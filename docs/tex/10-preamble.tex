% Добавляем гипертекстовое оглавление в PDF
\usepackage[
bookmarks=true, colorlinks=true, unicode=true,
urlcolor=black,linkcolor=black, anchorcolor=black,
citecolor=black, menucolor=black, filecolor=black,
]{hyperref}

% Изменение начертания шрифта --- после чего выглядит таймсоподобно.
% apt-get install scalable-cyrfonts-tex

\usepackage{graphicx}   % Пакет для включения рисунков
\DeclareGraphicsExtensions{.jpg,.pdf,.png}
% С такими оно полями оно работает по-умолчанию:
 \RequirePackage[left=20mm,right=10mm,top=20mm,bottom=20mm,headsep=0pt]{geometry}
% Если вас тошнит от поля в 10мм --- увеличивайте до 20-ти, ну и про переплёт не забывайте:
\geometry{right=20mm}
\geometry{left=15mm}



% Произвольная нумерация списков.

\usepackage{enumerate}

\setcounter{tocdepth}{1} 

\linespread{1}
\usepackage{rotating}
\usepackage{afterpage}

%\usepackage{fancyhdr}
%\fancyhead[C]{\thepage\\RU.\#\# ТЗ 01-1-ЛУ}
%\renewcommand{\headrulewidth}{0pt}