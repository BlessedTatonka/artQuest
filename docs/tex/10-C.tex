\pagestyle{fancy}

\subsection{Статус требований}

\begin{table}[h]
	\caption{Статус требований}
	\centering
	\begin{tabular}{|l|l|}
		\hline
		\textbf{Proposed}    & \begin{tabular}[c]{@{}l@{}}Требование запрошено авторизированным\\ источником.\end{tabular}                                                                                                                                                             \\ \hline
		\textbf{Approved}    & \begin{tabular}[c]{@{}l@{}}Требование проанализировано, его влияние\\ на проект просчитано, и оно было размещено\\ в базовой версии определенной версии.\end{tabular}                                                                                   \\ \hline
		\textbf{Implemented} & \begin{tabular}[c]{@{}l@{}}Код, реализующий требование, разработан,\\ написан и протестирован. Требование отслежено\\ до соответствующих элементов дизайна и кода.\end{tabular}                                                                         \\ \hline
		\textbf{Verified}    & \begin{tabular}[c]{@{}l@{}}Корректное функционирование реализованного\\ требования подтверждено в соответствующем продукте.\\ Требование отслежено до соответствующих\\ вариантов тестирования. Теперь требование\\ считается завершенным.\end{tabular} \\ \hline
		\textbf{Deleted}     & Утвержденное требование удалено из базовой версии.                                                                                                                                                                                                      \\ \hline
		\textbf{Rejected}    & \begin{tabular}[c]{@{}l@{}}Требование предложено, но не запланировано для\\ реализации ни в одной будущих версий.\end{tabular}                                                                                                                          \\ \hline
	\end{tabular}
\end{table}