\pagestyle{fancy}

\subsection{Список источников}

\begin{enumerate}
	\item[1] ГОСТ 19.101-77 Виды программ и программных документов. //Единая система программной документации. – М.: ИПК Издательство стандартов, 2001.
	\item[2] ГОСТ 19.106-78 Требования к программным документам, выполненным печатным способом. //Единая система программной документации. – М.: ИПК Издательство стандартов, 2001.
	\item[3] ГОСТ 19.301-79 Программа и методика испытаний. Требования к содержанию и оформлению. //Единая система программной документации. – М.: ИПК Издательство стандартов, 2001.
	\item[4] Developer apple. [Электронный ресурс]// URL: https://developers.google.com/ (Дата обращения: 01.12.2019, режим доступа: свободный).
	\item[5] Xamarin documentation. [Электронный ресурс]// URL: https://docs.microsoft.com/en-us/xamarin/ (Дата обращения: 26.11.2019, режим доступа: свободный).
	\item[6] ГОСТ 15150-69 Машины, приборы и другие технические изделия. Исполнения для различных климатических районов. Категории, условия эксплуатации, хранения и транспортирования в части воздействия климатических факторов внешней среды. – М.: Изд-во стандартов, 1997. 
	\item[7] Google developers. [Электронный ресурс]// URL: https://developers.google.com/ (Дата обращения: 02.12.2019, режим доступа: свободный).
	\item[8] ГОСТ 19.602-78 Правила дублирования, учета и хранения программных документов, выполненных печатным способом. //Единая система программной документации. – М.: ИПК Издательство стандартов, 2001.
\end{enumerate}