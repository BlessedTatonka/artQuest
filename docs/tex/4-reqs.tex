\pagestyle{fancy}

\begin{center}
		\section{ТРЕБОВАНИЯ К ПРОГРАММЕ}
\end{center}	

\subsection{Требования к функциональным характеристикам} 
	
	\subsubsection{Требования к составу выполняемых функций}	

		\hspace{10mm} Программа выполняется в рамках темы курсовой работы в соответствии с учебным
		планом подготовки бакалавров по направлению 09.03.04 «Программная инженерия»
		Национального исследовательского университета «Высшая школа экономики», факультет компьютерных наук.
		
		\hspace{10mm} Разработка требований велась совместно с командой и заказчиком в рамках предмета «Групповая динамика и коммуникации в программной инженерии»
		
		\textit{Состав команды:}
		\begin{itemize}
			\item Борис Малашенко -- android-developer;
			\item Тимофей Семенкович -- backend-developer;
			\item Дмитрий Сорокин -- iOS-developer;
			\item Дмитрий Штеменко -- android-developer;
		\end{itemize}
		
		\textit{Заказчик:}
		\begin{itemize}
			\item Авшалумова Бациин, Школа дизайна/ Кафедра «Коммуникационный дизайн».
		\end{itemize}

	\subsection*{FR-1. Авторизация клиента}

	\hspace{10mm}Чтобы использовать все возможности программы, клиент должен авторизоваться в системе
	
		\subsubsection*{FR-1.1.} При регистрации в программе  клиенту необходимо заполнить обязательные поля регистрации:
		
		\begin{enumerate}
			\item
				имя и фамилия;
			\item
				email или номер телефона;
			\item
				пароль.	
				
		\end{enumerate}
	
		\subsubsection*{FR-1.2.} 
		\hspace{10mm} Уже зарегистрированный клиент для входа в социальную сеть должен ввести
		свою почту или номер телефона и пароль.
	
		\subsubsection*{FR-1.3.}
		\hspace{10mm}После успешной регистрации пользователю будет выслан код
		на введенную им почту, который нужно ввести в специальное поле в приложении, только после правильного ввода кода клиент сможет войти в социальную
		сеть.

		\subsubsection*{FR-1.4.}
		\hspace{10mm} Если пользователь забыл пароль от системы, ему на почту/телефона должна быть выслана форма восстановления.
		
		\subsubsection*{FR-1.5.}
		\hspace{10mm} У пользователя должна быть возможность авторизоваться при помощи:
		\begin{enumerate}
			\item социальной сети Facebook;
			\item социальной сети Twitter;
			\item социальной сети Вконтакте.
		\end{enumerate}
		
		\subsection*{FR-2. Прохождение квестов}
		
		\subsubsection*{FR-2.1.}

		
		\hspace{10mm}У пользователя должна быть возможность просмотра актуальных квестов с основной информацией о них.
		
		\subsubsection*{FR-2.2.}
		
		
		\hspace{10mm}Должна быть возможность фильтровать список квестов по параметрам:
		
		\begin{enumerate}
			\item последние / популярные;
			\item сложность;		
			\item тема;	
			\item дистанция от пользователя до места проведения квеста;
			\item примерное время, за которое проходится выставка, связанная с квестом.
		\end{enumerate}
	
		\subsubsection*{FR-2.3.}
		
		\hspace{10mm} При выборе квеста пользователю должна быть предоставлена основная информация о квесте и выставке, связанной с ним. Должны отображаться следующие данные:
		\begin{enumerate}
		 	\item краткая информация о мероприятии;
		 	\item цена билета на мероприятие;
		 	\item возраст, с которого позволен вход на мероприятие;
		 	\item время прохождения мероприятия;
		 	\item сложность мероприятия;
		 	\item отзывы о мероприятии.
		\end{enumerate}
	
		\hspace{10mm} Если пользователь посещал связанную с квестом выставку, у него должна быть возможность пройти этот квест.
		
		\subsubsection*{FR-2.4.}
		
			\hspace{10mm} При прохождении вопросов квеста пользователь может использовать подсказку на данный вопрос, но потеряет часть своих баллов при верном ответе.
			
			\hspace{10mm} Пользователю должны предоставляться как минимум три вида вопросов:
			\begin{enumerate}
				\item с выбором ответа из списка;
				\item вида "собери картинку";
				\item вида "впиши верное слово".
			\end{enumerate}
		
		\subsubsection*{FR-2.5.}
		
			\hspace{10mm} После прохождения квеста у пользователя должна быть возможность оценить пройденный квест.
		
			\hspace{10mm} Кроме того пользователь сможет разместить запись о прохождении и оценке на личной странице приложения <<Арт-квест>>, либо на странице одной из своих социальных сетей.
			
		\subsubsection*{FR-2.6.}
		
			\hspace{10mm} При успешном прохождении квеста, пользователь должен получить бонусные баллы
			
			
	\subsection*{FR-3. Профиль}
	
			\hspace{10mm} У пользователя должна быть личная страница с основной информацией:
			\begin{enumerate}
				\item имя и фамилия пользователя
				\item пройденные квесты;
				\item количество баллов.
			\end{enumerate}
		
	\subsection*{FR-4. Таблицы лидеров}
	
			\hspace{10mm} В приложении должны быть таблицы лидеров за:
			\begin{enumerate}
				\item все время;
				\item последний месяц;
				\item последнюю неделю.
			\end{enumerate}
		
		\hspace{10mm} Пользователь должен видеть свое положение в каждой из таблиц.
		
	\subsection*{FR-5. Магазин наград}
	
		\hspace{10mm} Пользователь должен видеть все товары, которые он может приобрести за баллы.
		
	\subsection*{FR-6. Настройки}	
	
		\hspace{10mm} У пользователя должна быть возможность изменить свои:
		\begin{enumerate}
			\item фото;
			\item имя и фамилию;
			\item пароль;
			\item электронную почту;
			\item телефон.
		\end{enumerate}
	
		\hspace{10mm} У клиента должна быть возможность включить/выключить уведомления, звуковые эффекты.
		
		\hspace{10mm} Должна быть возможность поделиться успехами, если привязаны соц. сети, оставить отзыв программе, либо выйти из аккаунта.