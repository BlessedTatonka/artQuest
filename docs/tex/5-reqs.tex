\pagestyle{fancy}



	\section{ТРЕБОВАНИЯ К ПРОГРАММНОЙ ДОКУМЕНТАЦИИ}


\subsection{Предварительный состав программной документации} 
	
\normalsize
\begin{enumerate}
	\item <<Мобильное приложение <<Арт-квест>>. Техническое задание (ГОСТ 19.201-78);
	\item <<Мобильное приложение <<Арт-квест>>. Программа и методика испытаний (ГОСТ 19.301-78);
	\item <<Мобильное приложение <<Арт-квест>>. Текст программы (ГОСТ 19.401-78);
	\item <<Мобильное приложение <<Арт-квест>>. Пояснительная записка (ГОСТ 19.404-79);
	\item <<Мобильное приложение <<Арт-квест>>. Руководство оператора (ГОСТ 19.505-79); 
\end{enumerate}
	
\subsection{Специальные требования к программной документации}

	\begin{enumerate}
		\item Все документы к программе должны быть выполнены в соответствии с ГОСТ 19.106-78 [2] и ГОСТ к этому виду документа (см. п. 5.1.).
		\item Пояснительная записка должна быть загружена в систему Антиплагиат через ЛМС НИУ ВШЭ. Лист, подтверждающий загрузку пояснительной записки, сдается в учебный офис вместе со всеми материалами не позже, чем за день до защиты проектной работы.
		\item Вся документация сдается в печатном виде, при этом она должна быть обязательно подписана академическим руководителем образовательной программы 09.03.04 «Программная инженерия», руководителем разработки и исполнителем перед сдачей проектной работы в учебный офис не позже одного дня до защиты.
		\item Вся документация и программа также сдается в электронном виде в формате .pdf или .docx. в архиве формата .rar или .zip..
	\end{enumerate}
	

